\documentclass[12pt]{article}
\usepackage{fontspec}
\usepackage{polyglossia}
\usepackage{enumerate}
\usepackage{systeme}
\setdefaultlanguage{russian}
\setmainfont[Mapping=tex-text]{CMU Serif}
\usepackage{color}
\usepackage{amsmath}
\usepackage{amssymb,latexsym}

\begin{document}

В данной брошюре приведены условия задач, предлагавшихся на
вступительных испытаниях в 8 и 9 классы (в 9 класс набор производился
только в 2003, 2004 и 2006 годах) Лицея <<Физико-техническая школа>> при
ФТИ им.~А.~Ф.~Иоффе~РАН.

\section*{1988 год}
\begin{enumerate}
    \item Однажды в январе было 4 понедельника и 4 пятницы. Каким днем недели было 1 января?
    \item Можно ли из 20 монет достоинством 5, 20 и 50 копеек составить 5 рублей?
    \item Как разрезать квадрат на три части, из которых можно составить треугольник:
    \begin{enumerate}[a)]
        \item остроугольный,
        \item тупоугольный,
        \item прямоугольный,
    \end{enumerate}
\end{enumerate}

\section* {1989 год}
\begin{enumerate}
    \item $\frac{m}{n}$ \textemdash правильная дробь.
    Какое из чисел ближе к единице: $\frac{m}{n}$ или $\frac{n}{m}$?
    \item  Одна из сторон прямоугольника увеличилась на 25 \%, но его площадь
не изменилась. На сколько процентов изменилась другая сторона?
    \item Разделите фигуру, изображенную на рисунке, отрезком или ломаной на две равные части. Ответ поясните. {\color{red} TO-DO: add pic}
    \item Дробь $\frac{1}{7}$ обращается в десятичную. Какая цифра будет на сотом месте после запятой?
    \item Можно ли из 37 ниток сплести сетку так, чтобы каждая из них была сплетена ровно с пятью другими? 
    \item  $a^3 + b^3 + c^3 = 3abc$, $a \ne b$. Найдите $a + b + c$.
\end{enumerate}

\section* {1990 год}
\begin{enumerate}
    \item Разложите на множители:
$x^4+x^3+x^2+x-4$.
    \item Существуют ли такие натуральные числа $a$ и $b$ , что: $a^2 + 1990 = b^2$?
    \item Чему равен x , если: $(x^2+1)^4+(x^2+2)^2=5$?
    \item Можно ли сделать больше, чем 10, сумму дробей такого вида: $\frac12 + \frac14 + \frac16 + \frac18 + $...?
    \item Из бумаги сделали два одинаковых выпуклых четырехугольника. Пер-
вый из них разрезали по одной диагонали, второй – по другой. Можно
ли из четырех полученных треугольников составить параллелограмм?
    \item Решите неравенство: $|x+2| + |x-5|\geq7$. (Замечание: $|x|=x$, если $
    x\geq0$ и $|x|=-x$, если $x\le0$.)
    \item Расположите 10 точек на 5 отрезках так, чтобы на каждом отрезке было 4 точки. 
\end{enumerate}

\section*{1991 год}
\begin{enumerate}
\item  Положительное число увеличили на 20\%, а затем результат уменьшили
на 20\%. Во второй раз это же число сначала уменьшили на 20\%, а затем результат увеличили на 20\%. В какой раз получилось большее число? Решите задачу в общем виде.
\item Продолжите последовательность, дописав 2 числа: 13, 23, 43, 53, 73…
Какое число стоит на 566-м месте?
\item Назовем точку "целой", если обе ее координаты являются целыми чис-
лами.
\begin{enumerate}
\item[a)] Пусть известно, что на прямой $y = kx + 1$ одна целая точка есть. Есть
ли на ней еще хотя бы одна "целая" точка?
\item[б)] Может ли на прямой $y = kx + b$ быть ровно одна "целая" точка?
\end{enumerate}
\item Разложите на множители выражение:
$x^2 - \frac{1}{6}x - \frac{1}{6}$.
\item Прямые \textit{a} и \textit{b} параллельны. На прямой \textit{a} отмечено 24 точки. На прямой \textit{b} отмечено 25 точек.
\begin{enumerate}
\item[а)] Сколько можно провести отрезков с концами в отмеченных точках?
(Концы каждого отрезка лежат на разных прямых.)
\item[б)] Какое наибольшее число пересечений могут иметь все проведенные
отрезки? (Точки пересечения лежат внутри отрезков.) 
\end{enumerate}
\end{enumerate}


\section*{1992 год}

\begin{enumerate}
    \item В одном классе все любят бегать, прыгать и плавать. 60\% детей любят
    бегать, 60\% детей любят прыгать, 40\% детей любят плавать. Сколько
    процентов учеников любят все три занятия?
    \item Пусть при некотором значении $x$ выражение $x^3+x-2$ равно $0$. Чему
    равно при этом же значении $x$:
    \begin{enumerate}
        \item[a)] $x^6+4x-x^2-4$,  
        \item[б)] $x^3-x+2$?
    \end{enumerate}
    \item Выражение $1+x+x^2$ возвели в 6-ю степень, затем привели подобные
    члены и получили многочлен.
    \begin{enumerate}
        \item[a)] Чему равен коэффициент перед $x$?  
        \item[б)] Чему равна сумма всех коэффициентов многочлена?
    \end{enumerate}
    \item Можно ли сократить дробь $\frac{400003}{500004}$?
\end{enumerate}
\section*{1994 год}
\begin{enumerate}
\item В этой записи каждой буквой зашифрована какая-то цифра (разные буквы – разные цифры). Переведите эту запись с букв на цифры: КОШКА $+$ КОШКА $+$ КОШКА $=$ СОБАКА.
\item Поезд прошел расстояние от $A$ до $B$ и обратно.В одну сторону он шел со скоростью $V_1$, а в другую --- $V_2$. При этом 60 км$/$ч $<$ $V_1$ $<$ 70 км$/$ч, a 50 км$/$ч $<$ $V_2$ $<$ 60 км$/$ч. Оцените среднюю скорость поезда на всем пути от $A$ до $B$ и обратно с точностью до 1 км$/$ч.
\item Вычислите произведение: $\left(1-\frac{1}{2^2}\right)\cdot\left(1-\frac{1}{3^2}\right)\cdot\ldots\cdot\left(1-\frac{1}{2^2}\right)\cdot\left(1-\frac{1}{10^2}\right)$. Обобщите задачу.
\item Дан круг радиуса 2. На каждом нарисованном радиусе этого круга построен круг. Чему равны площади частей, на которые при этом разбился круг? 
\item Можно ли найти численное значение выражения $a^3 + b^3$, если \mbox{$a>0,b>0$} и
\begin{description}
\item[a)] $a^2+b^2 = 1$, $a+b=2$;
\item[b)] $a^2+b^2 = a + b$.
\end{description}

\end{enumerate}

\section*{1995 год}

\begin{enumerate}
     \item Пусть $x+y+z=0$. Упростите выражение $x^3+y^3+z^3-2xyz$.
     \item Чему равно число $\cfrac{1}{2-\cfrac{1}{2-\cfrac{1}{2-\ldots}}}$  (в такой записи числа 100 двоек)?
     \item Некто ходит в ФТШ на вечерние занятия. Туда он идет со скоростью $V_1$, а обратно --- той же дорогой, но со скоростью $V_2$. В каком из ниже
        перечисленных случаев его средняя скорость на всем пути туда и обратно может быть $5$ км/ч:
        \begin{enumerate}
            \item[a)] $V_1 > 4$ км/ч, $V_2 > 6$ км/ч;
            \item[б)] $V_1 < 4$ км/ч, $V_2 < 6$ км/ч;
            \item[в)] $V_1 > 4$ км/ч, $V_2 < 6$ км/ч?
        \end{enumerate}
        
    \item Три выпускника ФТШ, готовясь к поступлению на фи\-зи\-ко-тех\-ни\-чес\-кий факультет Технического университета, решили вместе 90
    задач. При этом каждый решил 60 задач. Задачи были легкими, средними и трудными. Легкая задача --- это задача, которую решили все
    трое, средняя --- та, которую решили ровно двое, а трудная --- та, которую решил только один из них.
    \begin{enumerate}
        \item[а)] Каких задач было больше: легких или трудных?
        \item[б)] Что вы можете сказать о числе средних задач?
    \end{enumerate}
     
\end{enumerate}

\section*{1996 год}

\begin{enumerate}
    \item Найдите \textit{a} , \textit{b} и \textit{c} , если известно, что:
    
    \systeme{a - b - c = 1, b - c - a = 2, c - a - b = 3} 
    
    \item Найдите наибольшее пятизначное число, у которого все цифры различны, первая цифра меньше второй, вторая меньше четвертой, а сум-
ма третьей и пятой цифр не больше первой. 

    \item $a$, $b$ и $c$ --- натуральные числа. Известно, что $ax^2+bx+c$  делится на 7 для любого натурального x . Докажите, что $a$, $b$ и $c$ делятся на 7. 
    \item Два угла расположены так, что их стороны имеют ровно 4 общие точки. Величина каждого угла равна $\alpha$. Какие-то 2 стороны этих углов
пересекаются под прямым углом. Под каким углом пересекаются их
биссектрисы? 
    \item Некоторые вершины правильного 13-угольника (т.е. 13-угольника, у
которого все стороны равны между собой и все углы равны между собой) покрашены в синий цвет, а остальные --– в красный. Докажите, что
найдется равнобедренный треугольник, все вершины которого лежат в
вершинах 13-угольника, покрашенных в один цвет. 
\end{enumerate}

\section*{1997 год}
\begin{enumerate}
\item  Докажите (без помощи калькулятора), что число
\begin{enumerate}
\item[a)] $1997\cdot1999 + 1$
\item[б)] $1995\cdot1997\cdot1999\cdot2001 + 16$
\end{enumerate}
является квадратом целого числа. 
\item Найдите наименьшее положительное число такое, что оно само является квадратом целого числа, его половина --- кубом целого числа, а его
треть --- пятой степенью целого числа. 
\item В примере на умножение все цифры заменили звездочками.
Восстановите пример, если известно, что все цифры –-- простые числа.  %%todo Вставить картинку%%
\item В треугольнике \textit{ABC} медиана и биссектриса, проведенные
из вершины \textit{A}, совпадают. Докажите, что \textit{ABC} –-- равнобедренный треугольник. 
\item Можно ли разрезать правильный шестиугольник на параллелограммы?
Можно ли разрезать правильный треугольник на параллелограммы?
(Фигура называется правильной, если все ее стороны и все ее углы
равны между собой.) 
\end{enumerate}

\section* {1998 год}
\begin{enumerate}
    \item У трехколёсного велосипеда все колёса разных размеров. Радиус второго колеса больше радиуса первого в 2 раза, а радиус третьего колеса
        в 3 раза больше радиуса первого. Велосипед проехал некоторое расстояние,
        причем все колёса вместе сделали 66 оборотов. Сколько оборотов сделало
        третье колесо? 
    \item Решите систему уравнений:
        $
        \begin{cases}
            a + b = 3\\
            a + c = 4\\
            a + d = 5\\
            b + c + d = 6
        \end{cases}
        $
    \item $\triangle ABC$ --- равнобедренный треугольник $(AB = BC)$. Точка $D$ на стороне $AC$ выбрана так, что треугольники $ADB$ и $BDC$ тоже оказались равнобедренными $(AB = BD = CD)$. Найдите $\angle ACB$.
    \item Сколько есть четных трехзначных чисел, больших 200, у которых вторая цифра больше первой, а третья \textemdash{} больше второй?
    \item Найдите все такие пары натуральных чисел $m$ и $n$, что $mn + 2 = 4m + 3n$.
    \item В треугольнике $ABC$ провели биссектрису $AD$. Известно, что $AB > AC$. Докажите, что $BD > CD$.
\end{enumerate}

\section*{1999 год}
\begin{enumerate}
\item Длина стороны квадрата $A_1B_1C_1D_1$ равна 8. Соединив середины соседних сторон этого квадрата, получаем квадрат $A_2B_2C_2D_2$. Затем ту же
процедуру проделали с квадратом $A_2B_2C_2D_2$ и получили квадрат
$A_3B_3C_3D_3$ и т.д. Что больше: площадь квадрата $A_7B_7C_7D_7$ или 1? 
\item Обозначим через $c(n)$ сумму цифр натурального числа $n$. Например,
$c(1998) = 27$, a $c(c(1998)) = c(27) = 9$. Найдите наибольшее восьмизначное число \textit{A} такое, что $c(c(A))= 2$. 
\item Дана таблица чисел. Можно ли изменить знаки ровно у двух чисел так,
чтобы после этого сумма всех чисел в получившейся таблице равнялась нулю:

a) 
$\begin{pmatrix}
  3 & 0 & 3\\
  -7 & 1 & 6\\
  1 & 1 & -2
\end{pmatrix}; $
\hspace{20pt}
б)
$\begin{pmatrix}
  6 & 2 & -5\\
  2 & -2 & -1\\
  1 & 1 & -1
\end{pmatrix} $
\item Имеется пять отрезков с длинами 2, 3, 4, 6, 7. Сколько различных треугольников можно составить из этих отрезков так, чтобы каждая сторона построенного треугольника являлась одним из данных отрезков? 
\item Петя, имея некоторое количество конфет, 1/7 часть отдал Сереже, а 1/4
часть --– Кате. Оставшиеся 17 конфет Петя съел сам. Сколько конфет
было у Пети? 
\item Решите уравнение: $\frac{x^3}{3} = 2x(x-2) + 3$ 
\item Сколько всего натуральных делителей у числа $91^{91}$? 
\end{enumerate}

\section* {2000 год}
\begin{enumerate}
\item Теплоход прошел по течению реки 224\textit{км} и столько же против течения, затратив на весь путь 15 часов. Скорость теплохода по течению 32 \textit{км/ч}. Найдите скорость течения реки. 
\item Три четвертых некоторого двузначного числа $n$ представляют собой натуральное число, делящееся на 7, а пять третьих от $n$ --- натуральное число, делящееся на 4. Найдите $n$.

\item В прямоугольном треугольнике ABC с катетами $AB=1$ и $AC=2$ из вершины прямого угла проведены высота AH, биссектриса AL и медиана AM. 
\begin{enumerate}
\item[a)] Докажите, что $AM=\frac12BC$.
\item[б)] Сравните по величине углы $\angle HAL$ и $\angle LAM$.
\end{enumerate}
\item Пусть $C(n)$ – сумма цифр натурального числа $n$ (например, \mbox{$C(1956)=21)$}. Существует ли такое натуральное число $n$ такое, что $C(n)$ и $C(n)$ делятся на 5? 
\item При каких значениях $a$ три отрезка, длины которых равны $5a-2$, $a$ и $4a+1$, образуют равнобедренный треугольник? 
\item Найдите наименьшее натуральное число $m$ такое, что наибольший общий делитель чисел $4m + 7$ и $7m + 4$ равен $33$. 
\end{enumerate}

\section*{2002 год}
\begin{enumerate}
    \item Решите уравнение $x^3+4x^2+5x+2=0$.
    \item Сколько существует двузначных чисел, у которых произведение цифр
    не больше 6?
    \item Докажите, что сумма дробей $\frac{2}{3}+\frac{4}{5}+\frac{6}{7}+\frac{8}{9}+\frac{10}{11}+\frac{3}{2}+\frac{5}{4}+\frac{7}{6}+\frac{9}{9}+\frac{11}{10}$ больше 10.
    \item В треугольнике $ABC$ провели медиану $BM$. Оказалось, что $BM=AM$.
    Найдите угол между высотой треугольника $ABM$, проведенной из точки $M$, и медианой треугольника $BMC$, проведенной из точки $M$.
    \item Какие значения может принимать число $x$, если для некоторого числа $a$ верно
    $x^2\left(4a^2+1\right)=4x\left(3a+1\right)-13$? Приведите все возможные варианты.
    \item На окружности отмечено 84 точки. Двое игроков играют в интересную
    игру. Они по очереди соединяют какие-то две точки отрезком так, чтобы он не пересекал проведенные раньше отрезки. Проигрывает тот, кто
    не сможет сделать ход. Докажите, что первый игрок может играть так,
    чтобы выиграть. 
\end{enumerate}

\end{document}
